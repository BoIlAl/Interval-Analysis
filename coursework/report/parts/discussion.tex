\documentclass[../body.tex]{subfiles}
\begin{document}
\begin{itemize}
    \item Субдифференциальный метод Ньютона является модификацией метода градиентного спуска, который, однако, не накладывает на отображение требования дифференцируемости, а требует лишь порядковой выпуклости
    \item Алгоритм дает точное решение задачи за небольшое конечное число итераций, обычно не превышающее размерность системы. В некоторых случаях параметр релаксации $\tau < 1$  улучшает сходимость субдифференциального метода Ньютона. Но чем меньше  $\tau$, тем медленнее работает Алгоритм.
    \item Кроме того, понижение параметра релаксации помогает выделить четкую "область сходимости" в ситуациях колебания. Операция не дает результата, если колебания нет.
\end{itemize}
\end{document}